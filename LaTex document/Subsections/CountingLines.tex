A part of the measurament process consist in counting lines of code. The way in which the counting process is performed is ruled by a standard methodology. The whole project is first formatted using the standard IDE autoformatting functionality. In Android Studio IDE, for example, the whole code can be autoformatted selecting the \textit{lib }folder and using CTRL+ALT+L shortcut. This passage allows the code to be uniformly distributed . Moreover, it allows to evaluate two identical pieces of code with the same result. After the formatting process lines are being counted manually line by line without taking into account empty lines. If a particular file is partially changed, with respect to the shared structure, only mutated lines are kept into account. Parenthesis are also taken into account (also if they take up an entire line). Generated code is \textbf{not} taken into account. It is counted and reported separately.