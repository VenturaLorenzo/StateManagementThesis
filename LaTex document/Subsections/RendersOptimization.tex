
This part of the development process aims to perform some optimizations in terms of UI renderings and memory consumption. In particular, the code will be refactored in order to use the least UI renders possible or ,in other words, to call the least \textit{build} method runs possible. The focus is on the TodoView and TodoItem widgets. The TodoView widget should be rendered again only after a structural change in the \textit{filteredTodos} list, where the filteredTodos list represent the list containing the todos matching the current filter. A structural change is intended as a mutation of the length of the list or a substitution of its internal elements. Basically, a structural change occurs when a new todo is added or removed from the list or when the filter changes. If the filteredTodos list's change concerns a single todo (e.g. when its internal state is changed using the checkbox or the update feature) it is considered a non-structural change. The main difference is that, a structural change, needs to rebuild the entire TodoView widget ,instead , a non-structural change can rebuild only a subpart (the particular TodoItem widget). When a structural change occurs, more than one TodoItem widget is affected and by the change ,the most convinient way to mutate them all consistently ,is to rebuild the entire TodoView widget. Moreover, adding , deleting and substituting a TodoItem (and consequently add/delete/substitute a child to the TodoView tree's node) is only feasible by the parent widget and not by widgets on the same tree level. A non-structural change ,instead, affects only a specific TodoItem/child and so, it is possibile to rebuild the single element only. Those optimizations are not really necessary in this scenario. The implemented application is ,indeed, very simple and do not need this kind of improvements at all. This is just an experiment in order to define which solution performs better at handling optimizations and to give an adjunctive prospective in the final comparison.